\documentclass[12pt, a4paper, simple]{eskdtext}

\usepackage{hyperref}
\usepackage{env}
\usepackage{styles/coursework_lst}
\usepackage{styles/coursework_content_menu}
\usepackage{styles/coursework_tables}
\usepackage{styles/coursework_pictures}
\usepackage{styles/coursework_ref}

% Код
\ESKDletter{}{К}{Р}
\def \courseworkDocTypeNum {81}
\def \courseworkDocVer {00}
\def \courseworkCode {\ESKDtheLetterI\ESKDtheLetterII\ESKDtheLetterIII.\courseworkStudentGroupName\courseworkStudentGroupNum.\courseworkStudentCard-0\courseworkDocNum~\courseworkDocTypeNum~\courseworkDocVer}

\def \courseworkDocTopic {ПОЯСНИТЕЛЬНАЯ ЗАПИСКА}

% Графа 1 (наименование изделия/документа)
\ESKDcolumnI {\ESKDfontIII \courseworkTopic \\ \courseworkDocTopic}

% Графа 2 (обозначение документа)
\ESKDsignature {\courseworkCode}

% Графа 9 (наименование или различительный индекс предприятия) задает команда
\ESKDcolumnIX {\courseworkDepartment}

% Графа 11 (фамилии лиц, подписывающих документ) задают команды
\ESKDcolumnXIfI {\fontsize{10}{10}\selectfont\courseworkStudentSurname~\courseworkStudentName}
\ESKDcolumnXIfII {\fontsize{8}{10}\selectfont\courseworkTeacherSurname~\courseworkTeacherName}
\ESKDcolumnXIfV {\fontsize{8}{10}\selectfont\courseworkTeacherSurname~\courseworkTeacherName}

\begin{document}
    \begin{ESKDtitlePage}
    \begin{center}
        \courseworkMinEdu \\
        \courseworkEdu \\
        \courseworkKaf \\
    \end{center}

    \vfill

    \begin{center}
        \courseworkTopic \\
    \end{center}

    \vfill

    \begin{center}
        \textbf{\courseworkDocTopic} \\
        ПО ДИСЦИПЛИНЕ \courseworkDiscipline \\
    \end{center}

    \vfill

    \begin{center}
        \courseworkCode \\
        Листов \pageref{LastPage} \\
    \end{center}

    \vfill

    \begin{flushright}
        \begin{minipage}[t]{.49\textwidth}
            \begin{minipage}[t]{.75\textwidth}
                \begin{flushright}
                    Руководитель

                    \hspace{0pt}

                    Выполнил

                    \hspace{0pt}

                    Консультанты:

                    по ЕСПД

                    Рецензент
                \end{flushright}
            \end{minipage}
        \end{minipage}
        \begin{minipage}[t]{.49\textwidth}
            \begin{flushright}
                \begin{minipage}[t]{.75\textwidth}
                    \courseworkTeacherName~\courseworkTeacherSurname

                    \hspace{0pt}

                    \courseworkStudentName~\courseworkStudentSurname

                    \hspace{0pt}

                    \hspace{0pt}

                    \courseworkTeacherName~\courseworkTeacherSurname

                    \courseworkReviewerName~\courseworkReviewerSurname
                \end{minipage}
            \end{flushright}
            
        \end{minipage}
    \end{flushright}

    \vfill

    \begin{center}
        \ESKDtheYear
    \end{center}
\end{ESKDtitlePage}


    \ESKDthisStyle{empty}
    лист с заданием
    \newpage
    \ESKDthisStyle{formII}
    
    \tableofcontents
    \paragraph{ПРИЛОЖЕНИЕ А. КОД ПРОГРАММЫ}
    \paragraph{ПРИЛОЖЕНИЕ Б. СХЕМА ПРОГРАММЫ}
    \newpage

    \section*{ВВЕДЕНИЕ}
    \phantomsection
    \addcontentsline{toc}{section}{ВВЕДЕНИЕ}

    \newpage

    \section{АНАЛИЗ ПОСТАВКИ ЗАДАЧИ}
    \subsection{Результаты обследования системы}

    \newpage
    
    \subsection{Обоснование необходимости создания системы}

    \newpage

    \subsection{Формирование рекомендаций по созданию системы}

    \newpage

    \subsection{Разработка, оценка и выбор варианта концепции системы}

    \newpage

    \subsection{Постановка задачи на создание системы}

    \newpage
    
    \section{ПРОЕКТИРОВАНИЕ СТРУКТУРЫ ПРИЛОЖЕНИЯ}
    
    \subsection{Разработка структуры системы}

    \newpage

    \subsection{Проектирование программного обеспечения системы}

    \newpage

    \subsection{Проектирование интерфейса пользователя с системой}
    
    \newpage

    \section{РАЗРАБОТКА АЛГОРИТМОВ ФУНКЦИОНИРОВАНИЯ И СТРУКТУР ДАННЫХ}

    \newpage
    
    \section{РЕАЛИЗАЦИЯ ПРИЛОЖЕНИЯ И РЕЗУЛЬТАТЫ ИСПЫТАНИЙ}
    \subsection{Кодирование программного обеспечения}

    \newpage
    
    \subsection{Интеграция программного обеспечения}

    \newpage

    \subsection{Тестирование приложения}

    \newpage

    \section*{ЗАКЛЮЧЕНИЕ}
    \phantomsection
    \addcontentsline{toc}{section}{ЗАКЛЮЧЕНИЕ}

    \newpage

    \begingroup
      \section*{СПИСОК ИСПОЛЬЗОВАННЫХ ИСТОЧНИКОВ}
      \phantomsection
      \addcontentsline{toc}{section}{СПИСОК ИСПОЛЬЗОВАННЫХ ИСТОЧНИКОВ}
    
      \renewcommand{\addcontentsline}[3]{}
      \renewcommand{\section}[2]{}
    
      \begin{thebibliography}{}
        \bibitem{GOST_1}
        ГОСТ 2.105-95.
        Единая система конструкторской документации (ЕСКД). 
        Общие требования к текстовым документам. 

        \bibitem{GOST_2}
        ГОСТ 19.504-79. 
        Единая система программной документации ЕСПД. Руководство программиста. 
        Требования к содержанию и оформлению.

        \bibitem{GOST_3}
        ГОСТ 19.701-90. ЕСПД. 
        Схемы алгоритмов, программ, данных и систем. 
        Обозначения условные и правила выполнения.

        \bibitem{GOST_4}
        ГОСТ 19.005-85. ЕСПД. 
        Р-схемы алгоритмов и программ. Обозначения условные графические и правила выполнения.

        \bibitem{GOST_5}
        ГОСТ 19.101-77. ЕСПД. 
        Виды программ и программных документов.

        \bibitem{GOST_6}
        ГОСТ 19.102-77. ЕСПД. 
        Стадии разработки.

        \bibitem{GOST_7}
        ГОСТ 19.103-77. ЕСПД.  
        Обозначения программ и программных документов. 

        \bibitem{GOST_8}
        ГОСТ 19.401-78. ЕСПД. 
        Текст программы. Требования к содержанию и оформлению.

        \bibitem{GOST_9}
        ГОСТ 19.402-78. ЕСПД. 
        Описание программы.

        \bibitem{GOST_10}
        ГОСТ 7.1-2003. Система стандартов по информации, библиотечному и издательскому делу. 
        Библиографическая запись. Библиографическое описание. 
        Общие требования и правила составления. 
      \end{thebibliography}
    \endgroup
    \newpage
\end{document}
