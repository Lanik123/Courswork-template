\documentclass[12pt, a4paper, simple]{eskdtext}

\usepackage{hyperref}
\usepackage{env}
\usepackage{styles/coursework_lst}
\usepackage{styles/coursework_content_menu}
\usepackage{styles/coursework_tables}
\usepackage{styles/coursework_pictures}

% Код
\ESKDletter{}{К}{Р}
\def \courseworkDocTypeNum {12}
\def \courseworkDocVer {00}
\def \courseworkCode {\ESKDtheLetterI\ESKDtheLetterII\ESKDtheLetterIII.\courseworkStudentGroupName\courseworkStudentGroupNum.\courseworkStudentCard-0\courseworkDocNum~\courseworkDocTypeNum~\courseworkDocVer}

\def \courseworkDocTopic {ТЕКСТ ПРОГРАММЫ}

% колонтитулы
\usepackage{fancybox, fancyhdr}
\fancypagestyle{plain}
{
    \renewcommand{\footrulewidth}{10pt}          % Толщина отделяющей полоски снизу
    \renewcommand{\headrulewidth}{10pt}          % Толщина отделяющей полоски сверху
    \fancyhead[C]{\hfill\courseworkCode\hfill\thepage} % Сверху по центру выводить код
    \fancyfoot{}                                % Очистить нижний колонтитул
}

% Графа 1 (наименование изделия/документа)
\ESKDcolumnI {\ESKDfontIII \courseworkTopic \\ \courseworkDocTopic}

% Графа 2 (обозначение документа)
\ESKDsignature {\courseworkCode}

% Графа 9 (наименование или различительный индекс предприятия) задает команда
\ESKDcolumnIX {\courseworkDepartment}

% Графа 11 (фамилии лиц, подписывающих документ) задают команды
\ESKDcolumnXIfI {\fontsize{10}{10}\selectfont\courseworkStudentSurname~\courseworkStudentName}
\ESKDcolumnXIfII {\fontsize{8}{10}\selectfont\courseworkTeacherSurname~\courseworkTeacherName}
\ESKDcolumnXIfV {\fontsize{8}{10}\selectfont\courseworkTeacherSurname~\courseworkTeacherName}

\begin{document}
    \begin{ESKDtitlePage}
    \begin{flushright}
        \textbf{ПРИЛОЖЕНИЕ А} \enspace\enspace
    \end{flushright}
    \begin{center}
        \courseworkEdu \\
        \courseworkKaf \\
    \end{center}

    \vfill

    \begin{center}
        \courseworkTopic \\
    \end{center}

    \vfill

    \begin{center}
        \textbf{\courseworkDocTopic} \\
    \end{center}

    \vfill

    \begin{center}
        \courseworkCode \\
        Листов \pageref{LastPage} \\
    \end{center}

    \vfill

    \begin{flushright}
        \begin{minipage}[t]{.49\textwidth}
            \begin{minipage}[t]{.75\textwidth}
                \begin{flushright}
                    Руководитель

                    Выполнил

                    Консультант

                    по ЕСПД
                \end{flushright}
            \end{minipage}
        \end{minipage}
        \begin{minipage}[t]{.49\textwidth}
            \begin{flushright}
                \begin{minipage}[t]{.75\textwidth}
                    \courseworkTeacherName~\courseworkTeacherSurname

                    \courseworkStudentName~\courseworkStudentSurname

                    \hspace{0pt}

                    \courseworkTeacherName~\courseworkTeacherSurname

                \end{minipage}
            \end{flushright}
            
        \end{minipage}
    \end{flushright}

    \vfill

    \begin{center}
        \ESKDtheYear
    \end{center}
\end{ESKDtitlePage}


    \ESKDstyle{title}
    \thispagestyle{plain}
    \pagestyle{plain}
    \hspace{0pt}

    \paragraph{} \textbf{Исходный код приложения}
\end{document}
